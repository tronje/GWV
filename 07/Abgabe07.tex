\documentclass[12pt,a4paper]{article}
\usepackage[utf8]{inputenc}
\usepackage[english]{babel}
\usepackage[T1]{fontenc}
\usepackage{amsmath}
\usepackage{amsfonts}
\usepackage{amssymb}
\usepackage{svg}
\usepackage[hidelinks]{hyperref}

\author{Finn-Lasse Jörgensen, Frederik Wille, Tronje Krabbe}
\title{Tutorial 7: Constraint Satisfaction}
\begin{document}
\maketitle

\section*{Exercise 7.1 (Constraint)}
\begin{itemize}
\item
Jeder Buchstabe wird durch eine Variable mit der Domäne 0-9 beschrieben.
Jede Spalte ist ein Constraint zwischen den drei Buchstaben der Spalte, wobei die Summe der beiden Oberen Buchstaben kongruent zu dem Wert des Unteren Buchstaben modulo 10 sein muss.
In den Constraints muss zusätzlich bedacht werden, dass in der vorherigen Spalte ein Übertrag entstanden sein kann, den wir durch ein $OR$ in das Constraint einbringen.\\
\begin{figure}[ht]
    % \includegraphics[width=\textwidth]{gwv07_network.png}
    \includegraphics[scale=0.46]{gwv07_network.png}
    \caption{Constraint Network}
    \label{network}
\end{figure}
\end{itemize}

\end{document}
