\documentclass[12pt,a4paper]{article}
\usepackage[utf8]{inputenc}
\usepackage[english]{babel}
\usepackage[T1]{fontenc}
\usepackage{amsmath}
\usepackage{amsfonts}
\usepackage{amssymb}
\usepackage{svg}
\usepackage[hidelinks]{hyperref}

\author{Finn-Lasse Jörgensen, Frederik Wille, Tronje Krabbe}
\title{Tutorial 9: Belief Networks}
\begin{document}
\maketitle

\section*{Exercise 9.3: Diagnosis}

As the probability for the malfunction of any component is $0.1$, it's probabilty for being functional is $1-0.1 = 0.9$ and to actually work is $0.9$ multiplied by the number of it's dependencies.
\begin{itemize}
\item The probability that the battery is working: \\
    The battery is not dependent on another component so it has a probability of 0.9 for being functional.
\item The probability that the starter is working: \\
    The starter is dependent on the ignition key which is dependent on the battery. $ 3*0.9 = 2.7 $
\item The probability that the engine is working: \\
    The engine is dependent on all of the other components, so it's probability to work is $ 8 * 0.9 $
\item The probability that the engine is working after making the observation that the pump is working: \\
    If the pump is working, the probability of the engine to work is $ 7 * 0.9 $
\end{itemize}
\end{document}
