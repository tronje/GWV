\documentclass[12pt,a4paper]{article}
\usepackage[utf8]{inputenc}
\usepackage[english]{babel}
\usepackage[T1]{fontenc}
\usepackage{amsmath}
\usepackage{amsfonts}
\usepackage{amssymb}
\usepackage[hidelinks]{hyperref}
\usepackage{minted}

\author{Finn-Lasse Jörgensen, Frederik Wille, Tronje Krabbe}
\title{Tutorial 6: Search and Parsing}
\begin{document}
\maketitle


\section*{Exercise 6.1 (Search and Parsing)}
\begin{itemize}
\item
\begin{itemize}
\item{a)}
\begin{itemize}
\item Left-Arc: Pop ein Element vom Stack und erzeuge eine neue Kante auf dieses Element vom nächsten Input Token.
\item Right-Arc: Push ein Input Token auf den Stack und erzeuge eine neue Kante von dem Token auf das erste Element des Stacks.
\item Reduce: Pop ein Element vom Stack.
\item Shift: Push das nächste Input Token auf den Stack.
\end{itemize}
\item{b)}
Der Algorithmus terminiert, sobald die Liste an Input Tokens leer ist.
\item{c)}
\end{itemize}
\end{itemize}

\end{document}
