\documentclass[12pt,a4paper]{article}
\usepackage[utf8]{inputenc}
\usepackage[english]{babel}
\usepackage[T1]{fontenc}
\usepackage{amsmath}
\usepackage{amsfonts}
\usepackage{amssymb}

\author{Finn-Lasse Jörgensen, Frederik Wille, Tronje Krabbe}
\title{Tutorial 2: Search Spaces}
\begin{document}
\maketitle


\section*{Exercise 2.3: (Search Space Construction 2)} % electric boogaloo
Das Spielfeld kann als gerichteter Graph repräsentiert werden,
wobei Knoten, auf denen sich ein Spieler befindet, besonders
markiert sind.

Eine Möglichkeit, das Problem zu lösen, wäre alle möglichen Zielpunkte
von Mister X dahingehend zu überprüfen, ob die Kriminalbeamten
es dort hin schaffen. Dies könnte z.B. mithilfe von Dijkstras
Algorithmus geschehen. Wenn die Position aller Detektive zu einem
Zielpunkt von Mister X eine größere Minimaldistanz hat,
als Schritte erlaubt sind, so ist dieser ein guter Zug.

Diese Strategie ist wohl aber wahrscheinlich recht langsam.
Eine andere Herangehensweise wäre, alle Knoten, die von
den Kriminalbeamten erreichbar sind, zu markieren, z.B.
per einer Variation der Breitensuche, die alle Knoten, die sie abgeht,
markiert, und abbricht, wenn die Distanz größer ist, als
die erlaubte Anzahl Schritte.

Jetzt sucht man für Mister X, z.B. per Tiefensuche,
einen Knoten der nicht markiert, und natürlich auch nicht zu
weit weg ist.

\section*{Excercise 2.4: (Search Space Construction 3)}
\subsection*{Placing furniture in a flat}
Der Suchraum wird durch alle möglichen Platzierungen der Möbel gebildet.
Der Start der Suche ist ohne ein platziertes Möbelstück und Kanten zu
anderen Zuständen entstehen durch Platzieren eines weiteren Mobelstückes.
Ziel der Suche ist eine (von mehreren) für den Bewohner optimale Platzierung.
Wenn man davon ausgeht, dass nur eine begrenzte Anzahl an Möbeln
zur Verfügung steht, ist der Suchraum begrenzt. Dadurch, dass wir
nur Möbel in die Wohnung platzieren, entsteht ein gerichteter Graph ohne Zyklen.
Da wir nur eine von mehreren möglichen optimalen Lösungen suchen
und einen endlichen, zyklen-freien Graphen haben, bietet sich eine Tiefensuche an.

\subsection*{Construction Site planning}
Am einfachsten wäre es, die einzelnen Konstruktions-Schritte als Knoten
eines Baumes darzustellen. Dies ist leider in den meisten Fällen nicht
möglich, da einige Schritte mehrere Abhängigkeiten haben könnten.
Es wird also ein gerichteter Graph benötigt, wobei dieser wahrscheinlich
einem Baum ähnlich sehen würde. Er wird außerdem keine Zyklen enthalten,
da wohl kaum ganze Schritte rückgängig gemacht werden.

Damit das Haus fertig gebaut ist, müssen alle Schritte abgeschlossen
sein. Also muss man eine Knotenfolge finden, die vom Startknoten
zum Endknoten verläuft, und in der die Knoten in einer Reihenfolge angeordnet sind,
die keine Konflikte enthält.

Dies kann bewerkstelligt werden, indem man den Startknoten
zu einer Queue hinzufügt, und von dort aus eine Breitensuche macht.
Jeder Knoten, der gefunden wird, und noch nicht in der Struktur
enthalten ist, wird genau dann hinzugefügt, wenn alle seine Eltern-Knoten
bereits enthalten sind.
Erreicht die Breitensuche einen Status, in dem es nicht mehr weiter geht,
muss von vorne angefangen werden; so lange, bis alle Knoten in der Queue
eingeordnet sind. Danach kann das Haus entsprechend der Reihenfolge der Schritte
in der Queue gebaut werden.

\subsection*{Elevator}


\end{document}
