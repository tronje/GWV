\documentclass[12pt,a4paper]{article}
\usepackage[utf8]{inputenc}
\usepackage[english]{babel}
\usepackage[T1]{fontenc}
\usepackage{amsmath}
\usepackage{amsfonts}
\usepackage{amssymb}
\usepackage[hidelinks]{hyperref}

\author{Finn-Lasse Jörgensen, Frederik Wille, Tronje Krabbe}
\title{Tutorial 4: Searching}
\begin{document}
\maketitle


\section*{Exercise 4.2}
\subsection*{3.}
Die vorhandenen Beispiel-Labyrinthe zeigen bereits Unterschiede zwischen
Breiten- und Tiefensuche. \textit{blatt3\_environment.txt} beispielsweise
zeigt sehr deutlich, wie die Tiefensuche nach den Prioritäten der Nachfolger-Knoten
(bei uns momentan: erst oben, dann rechts, unten, und schliesslich links)
das Feld durchläuft, bis es irgendwann den Zielknoten findet.
Die Breitensuche läuft ebenfalls viele Knoten ab,
aber praktisch auf mehrere Pfade verteilt, sodass, findet sie den Zielknoten,
der resultierende Pfad sehr viel kürzer ist.

\subsection*{4.}
Da wir loop-detection haben, finden unsere Suchen auch immer zum Ziel.
Natürlich unter der Vorraussetzung, dass das Ziel überhaupt erreichbar ist.

\end{document}
