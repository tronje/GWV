\documentclass[10pt]{beamer}
\usepackage[german,ngerman]{babel}
\usepackage[utf8]{inputenc}


\usetheme[progressbar=frametitle]{metropolis}
\usepackage{appendixnumberbeamer}

\usepackage{booktabs}
\usepackage[scale=2]{ccicons}

\usepackage{pgfplots}
\usepgfplotslibrary{dateplot}

\usepackage{xspace}
\newcommand{\themename}{\textbf{\textsc{metropolis}}\xspace}

\usepackage{fancyvrb}

\usepackage{minted}

\usepackage{marvosym}

\title{Cryptoarithmetisches Puzzle}
\subtitle{GWV WiSe 2016/2017}
\date{\today}
\date{}
\author{Finn-Lasse Jörgensen, Frederik Wille, Tronje Krabbe}
\institute{UHH}
% \titlegraphic{\hfill\includegraphics[height=1.5cm]{logo.pdf}}

\begin{document}

\maketitle

% ToC? Don't really need it...
% \begin{frame}{Table of contents}
% \setbeamertemplate{section in toc}[sections numbered]
% \tableofcontents[hideallsubsections]
% \end{frame}

\section{Intro}

\begin{frame}[fragile]{Intro}

\begin{BVerbatim}
        THREE
      + THREE
      +  FIVE
      =ELEVEN
\end{BVerbatim}

\end{frame}

\begin{frame}[fragile]{Probleme}
    \begin{itemize}
        \item AC3 Implementieren
        \item Gute Constraints \& Variablen finden
    \end{itemize}
\end{frame}

\section{AC3-Implementation}

\begin{frame}[fragile]{AC3-Implementation}
    \begin{minted}{python}
class Network(object):

    def __init__(self, variables):
        self.variables = variables
        self._arcs = None
        self.constraints = []
        self.unary_constraints = []
    \end{minted}
\end{frame}

\begin{frame}[fragile]{AC3-Implementation}
    \begin{minted}{python}
class Variable(object):

    def __init__(self, domain=None, meta={}):
        if domain is None:
            self.domain = set()
        else:
            self.domain = domain
        self.meta = meta
        self.constraints = []
        self.arcs = set()
    \end{minted}
\end{frame}

\begin{frame}[fragile]{AC3-Implementation}
    \begin{minted}{python}
class Constraint(object):

    def __init__(self, variables, cfunc):
        self.variables = variables
        self.cfunc = cfunc

    def is_satisfied(self, values):
        assert type(values) is list
        return self.cfunc(*values)
    \end{minted}
\end{frame}

\begin{frame}[fragile]{AC3-Implementation}
    \begin{minted}{python}
class UnaryConstraint(object):

    def __init__(self, variable, cfunc):
        self.variable = variable
        self.cfunc = cfunc

    def is_satisfied(self, value):
        return self.cfunc(value)
    \end{minted}
\end{frame}

\begin{frame}[fragile]{AC3-Implementation}
    \begin{minted}{python}
class Arc(object):

    def __init__(self, variable, constraint):
        self.variable = variable
        self.constraint = constraint
    \end{minted}
\end{frame}

\begin{frame}[fragile]{AC3-Implementation}
    \begin{minted}[fontsize=\footnotesize]{python}
class Arc(object):
    # ...

    def make_consistent(self):
        non_consistent = []
        other_variable = self.other_variable()

        # find non-consistent elements
        for elem in self.variable.domain:
            for other in other_variable.domain:
                if self.constraint.is_satisfied([elem, other]):
                    break
            else:
                # for's else is executed if the loop
                # was *not* interrupted by a break!
                non_consistent.append(elem)

        # remove non-consistent elements from domain
        for elem in non_consistent:
            self.variable.domain.remove(elem)
    \end{minted}
\end{frame}

\begin{frame}[fragile]{AC3-Implementation}
    \begin{minted}[fontsize=\footnotesize]{python}
class Network(object):
    # ...
    def gac(self):
        todo_arcs = self.arcs

        self.satisfy_unary_constraints()

        while len(todo_arcs) > 0:
            arc = todo_arcs.pop()

            if not arc.is_consistent():
                arc.make_consistent()

                for variable in self.variables:
                    for narc in variable.arcs:
                        if (narc.other_variable() == arc.variable
                           and narc.constraint != arc.constraint):
                            todo_arcs.add(narc)
    \end{minted}
\end{frame}

\section{Constraints \& Variablen}

\begin{frame}[fragile]{Variablen}
    \begin{itemize}
        \item Pro Buchstabe: eine Variable
        \item Pro Spalte jeweils: eine Überlauf-Variable \& eine Summen-Variable
    \end{itemize}
\end{frame}

\begin{frame}[fragile]{}

\begin{BVerbatim}
        THREE
      + THREE
      +  FIVE
      =ELEVEN
\end{BVerbatim}

Variablen: $ {E, F, H, I, L, N, R, T, V} $ plus je sechs Überlauf- und Summen-Variablen

\end{frame}

\begin{frame}[fragile]{Constraints}
    \begin{itemize}
        \item Jede Buchstaben-Variable, deren Buchstabe irgendwo an erster Stelle steht:
            Ein \texttt{UnaryConstraint}, dass ihr Wert nicht 0 sein darf.
        \item Jede Buchstaben-Variable mit jeder anderen Buchstaben-Variable:
            Ein \texttt{Constraint}, dass die Werte ungleich sein müssen.
        \item Jede Summen-Variable: muss gleich der Variable unter dem Strich sein
            \Lightning
    \end{itemize}
\end{frame}

\begin{frame}[standout]
    Questions?
\end{frame}

\appendix

\begin{frame}[allowframebreaks]{References}

    \bibliography{demo}
    \bibliographystyle{abbrv}
    \url{http://artint.info/}\\
    Poole, Mackworth

\end{frame}

\end{document}

