\documentclass[12pt,a4paper]{article}
\usepackage[utf8]{inputenc}
\usepackage[english]{babel}
\usepackage[T1]{fontenc}
\usepackage{amsmath}
\usepackage{amsfonts}
\usepackage{amssymb}
\usepackage[hidelinks]{hyperref}
\usepackage{minted}

\author{Finn-Lasse Jörgensen, Frederik Wille, Tronje Krabbe}
\title{Tutorial 5: Searching}
\begin{document}
\maketitle


\section*{Exercise 5.2 (Heuristic Search)}
\subsection*{1.}
Unsere Heuristik-Funktion misst einfach die Distanz zum Ziel in der `Luftlinie'.
In einem 2D Labyrinth gibt es nicht viele Möglichkeiten, eine gute Heuristik aufzustellen.

\subsection*{2.}
Damit die Heuristik Portale unterstützt, muss sie eigentlich nur das Minimum der Distanz
des gegebenen Portal-Knotens und die seines Partner-Knotens (plus eins) zurückgeben:

\begin{minted}[fontsize=\footnotesize]{python}
def h_distance_portals(pfield, node, goal='g'):
    gnode = pfield.find_node(goal)

    # see if we've got a portal
    if node.is_portal():
        # find the partner node
        pnode = pfield.find_portal_exit(node)

        # distance when going through the portal
        pdist = abs(pnode.x - gnode.x) + abs(pnode.y - gnode.y) + 1 

        # distance without using the portal
        dist = abs(node.x - gnode.x) + abs(node.y - gnode.y)

        # return the minimum
        return min(dist, pdist)

    # default behaviour
    return abs(node.x - gnode.x) + abs(node.y - gnode.y)
\end{minted}

\subsection*{3.}
Wenn alle Knoten erkundet wurden, und noch kein Ziel gefunden wurde, bricht unser Programm mit einer Exception ab.

\subsection*{4.}

\subsection*{6.}
Hierfür mussten wir nichts weiter ändern, das Programm funktioniert standardmäßig so.

\end{document}
